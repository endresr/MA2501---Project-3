\documentclass[12pt, a4paper,usenames,dvipsnames]{article}
\usepackage[utf8]{inputenc}
\usepackage[english]{babel}
\usepackage{amsmath,amssymb,amsthm} 
\usepackage{graphicx}
\usepackage{siunitx}
    \sisetup{exponent-product = \cdot} 
    \sisetup{separate-uncertainty = true}
\usepackage[margin=20mm, tmargin=30mm,headheight=15pt ]{geometry}
\usepackage{fancyhdr}
\usepackage{framed}
\usepackage{caption}
\usepackage{subcaption}
\usepackage{lastpage}  
\pagestyle{fancy}
    \fancyhf{}
    \rhead{Project 3}
    \lhead{MA2501}
    \rfoot{Page \thepage \ of \pageref{LastPage}}
\fancypagestyle{noheader}{
  \fancyhf{}% Clear header/footer
  \renewcommand{\headrulewidth}{0pt}% No header rule
  \rfoot{Page \thepage \ of \pageref{LastPage}}
}
\usepackage{tikz}
\usetikzlibrary{calc}
\usetikzlibrary{patterns}
\usetikzlibrary{positioning}
\usepackage{lettrine}
\title{Project 3 - MA2501}
\author{Anna Bakkebø\And Thomas Schjem \And Endre Sørmo Rundsveen}
\usepackage{sectsty}
\usepackage{bm}


\definecolor{XKCDpale}{RGB}{255,249,208}
\definecolor{XKCDcrimson}{RGB}{140,0,15}
\definecolor{XKCDpalegray}{RGB}{253,253,254}

\sectionfont{\color{XKCDcrimson}}
\subsectionfont{\color{XKCDcrimson}}
\usepackage{listings}%Code show
\lstdefinestyle{mystyle}{
    backgroundcolor=\color{XKCDpalegray},
    commentstyle=\color{YellowGreen},
    keywordstyle=\color{XKCDcrimson}\bfseries,
    numberstyle=\tiny,
    stringstyle=\color{OliveGreen},
    basicstyle=\footnotesize,
    breakatwhitespace=false,         
    breaklines=true,                 
    captionpos=t,                    
    keepspaces=true,                 
    numbers=left,                    
    numbersep=1pt,                  
    showspaces=false,                
    showstringspaces=false,
    showtabs=false,                  
    tabsize=1,
    xleftmargin=0.5em,
    frame=topline
}
 
\lstset{style=mystyle}
\renewcommand\vec{\mathbf}

\begin{document}

\begin{titlepage}
    \newgeometry{margin=20mm,tmargin=110mm}
    \begin{tikzpicture}[remember picture,overlay]
        \fill[color=XKCDpale] ($(current page.south west)+(0cm,15cm)$) rectangle (current page.north east);
        \foreach \x in {2,3,4,5,6,7,8}
            {\fill[gray!40] (\x cm, 1)--(\x cm, {3+1.15*sin(0.5*pi*\x r)*exp(-\x/4)})--(\x +1,{3+1.15*sin(0.5*pi*(\x+1) r)*exp(-(\x+1)/4)})--(\x +1,1) --cycle;
            \draw[dashed] (\x cm, {3+1.15*sin(0.5*pi*\x r)*exp(-\x/4)})--(\x +1,{3+1.15*sin(0.5*pi*(\x+1) r)*exp(-(\x+1)/4)});}
        \draw[thick,->] (-2,1)--(15,1);
        \draw[thick,->] (-2,1)--(-2,8);
        \foreach \x in {2,3,4,5,6,7,8,9}
            {\draw (\x cm,1cm+2pt)--+(0,-4pt);
            \draw[dashed] (\x ,1)--(\x,{3+1.15*sin(0.5*pi*\x r)*exp(-\x/4)});}
        \node[anchor=north] at (2,1){\(a\)};
        \node[anchor=north] at(9,1){\(b\)};
        \draw[Mahogany, thick, domain=2:9] plot[samples=100] (\x, {3+1.15*sin(0.5*pi*\x r)*exp(-\x/4)});
        
    \end{tikzpicture}
  
    {\noindent \Huge \color{Brown}{\underline{Project 3 - MA2501}}}\\
    
    {\noindent\large \color{Brown}{Anna Bakkebø\\Thomas Schjem\\Endre Sørmo Rundsveen}}\\
    \raggedright
    \hfill \break
    \lettrine[lraise=0.15]{T}{} his document is the report for our third project in Numerical Methods - MA2501 the spring semester of 2019. The project mainly focuses on numerical integrations techniques and are based on the Newton-Cotes quadrature formula. First we familiarize ourself with the adaptive simpson method, before we continue with the Romber method, and lastly we end by exploring both the Runge-Kutta implicit integration method and the improved Euler method. \\ 
    \hspace{10pt}Note that we use some place to talk about the theory behind the methods. This is mainly for our own sake so we can easily consult this document at a later time to repeat. For readers well-versed in the underlying theory, these sections can be quite unnecessary, and can thus be skipped. 
    % \begin{lstlisting}[language=Python]
    % import matplotlib
    % import matplotlib.pyplot as plt
    % import numpy as np
    % import sumpy as symp\end{lstlisting}
\end{titlepage}
\restoregeometry
\twocolumn

\section*{Problem 1}
\subsection*{a)}
\subsubsection*{Theory}
The composite Simpson method for integrating a function 
\[\int_a^bf(x)dx\] 
divide the interval \([a,b]\) in \(n\) subintervals\\ \([x_{i-1},x_i]\), \(i=1,\cdots,n\), which it runs the Simpson method on, where \((x_i)_{i=0}^{i=n}\) is equidistant nodes \[a=x_0<x_1<\cdots<x_{n-1}<x_n=b\]

This will generally give a better approximation than to only run the Simpson method over the original interval, but it happens that a finer partition of an interval doesn't result in better approximation, e.g. if the function is a polynomial of grade 3 or lower the Simpson method will give the exact solution. In these situations, it is superfluous too further divide the interval and thus do more calculations. One would therefore prefer if the method could decide on whether it has reached a sufficient fine partition or not.

This leads up to the Adaptive Simpson \\Method. We take a general subinterval \([a_j,b_j]\) and use Simpson to get an estimate \(I_0=S(a_j,b_j)\) and then use the composite Simpson method on it with two subintervals to get the estimate \(I_1\). If \(I_0\) and \(I_1\) is sufficiently close, we can conclude that \(I_0\) is reasonably close, so any further division of the subinterval will be superfluous. Otherwise the subinterval is divided into two smaller subintervals which the same procedure is done on. This gives a recursive formula. 

To give a criteria for when \(I_0\) and \(I_1\) is sufficiently close, we look at the relationship between the error bound for Simpson on one interval and composite Simpson with two subintervals. The absolute error for the Simpson method over \([a_j,b_j]\) (\(\varepsilon_{[a_j,b_j]}\)) is bounded by 
\[\varepsilon_{[a_j,b_j]}\leq \frac{(b_j-a_j)^5}{2880}M_4,\quad M_4=\max_{\eta\in(a_j,b_j)}|f^{(4)}(\eta)|\]
If we use the simpson method on the two subintervals \([a_j,c_j],[c_j,b_j]\) where \(c_j=\frac{a_j+b_j}{2}\), we can see that the sum of the two errors is bounded by 
\begin{equation*}
    \begin{split}
        \varepsilon_{[a_j,c_j]}+\varepsilon_{[c_j,b_j]}\leq& \frac{(c_j-a_j)^5}{2880}M_4+\frac{(b_j-c_j)^5}{2880}M_4\\
        =&2\cdot\left(\frac{b_j-a_j}{2}\right)^5\cdot\frac{M_4}{2880}\\
        =&\frac{1}{16}\cdot\frac{(b_j-a_j)^5}{2880}M_4
    \end{split}
\end{equation*}
where \(M_4=\max_{\eta \in (a_j,b_j)}|f^{(4)}(\eta)|\). Hence the difference in error for the composite Simpson method with two subintervals on \([a_j,b_j]\) and Simpson method on the same interval is
\begin{equation*}
    \begin{split}
        \varepsilon_{[a_j,b_j]}-\varepsilon_{[a_j,c_j]}-\varepsilon_{[c_j,b_j]}\approx \frac{15}{16}\cdot\frac{(b_j-a_j)^5}{2880}M_4
    \end{split}
\end{equation*}
Using 
\[\int_a^bf(x)dx\approx S(a,b)+\varepsilon_{[a,b]}\]
we can rewrite this as
\[\frac{16}{15}\left(S(a_j,b_j)-S(a_j,c_j)-S(c_j,b_j)\right)\approx\frac{(b_j-a_j)^5}{2880}M_4\]
This result in
\begin{equation*}
    \begin{split}
        \int_{a_j}^{b_j}f(x)dx-I_1\approx&\frac{1}{16}\cdot\frac{(b_j-a_j)^5}{2880}M_4\\
         \approx&\frac{1}{15}\left(S(a_j,b_j)-S(a_j,c_j)-S(c_j,b_j)\right)\\
         =&\frac{1}{15}(I_0-I_1)
    \end{split}
\end{equation*}
Finally
\[\left|\int_{a_j}^{b_j}f(x)dx-I_1\right|\approx\frac{1}{15}\left|I_0-I_1\right|\]
That will say that if \(\frac{1}{15}|I_0-I_1|<TOL\) then \(I_1\) is within the tolerance limit set, over that subinterval. Therefore we can use this as en estimate of whether \(I_0\) and \(I_1\) is sifficiently close. For each time we divide a subinterval into two smallet subintervals, we half the tolerance for the method on those intervals.  
\subsection*{b)}

\begin{table}[h]
\begin{tabular}{|l|l|l|l|}
\hline
$\textbf{f(x)}$                     & \textbf{Exact}               & \textbf{Estimate} & \textbf{Error} \\ \hline
$cos(2 \pi x)$, $x \in [0, 1]$      & 0                            & Sett inn          & Sett inn       \\ \hline
$e^{3x}sin(2x)$, $x \in [0, \pi/4]$ & $\frac{1}{13}(2+3e^{3\pi/4})$ & Sett inn          & Sett inn       \\ \hline
\end{tabular}
\end{table}
\subsection*{c)}
\newpage
gbbr
reeg3wgergwrgsgz
\thispagestyle{noheader}
\onecolumn
\newgeometry{margin=20mm, tmargin=0.61\paperwidth,headheight=15pt}
\refstepcounter{figure}
\label{fig:graph}
\begin{tikzpicture}[remember picture,overlay]
    \fill[color=XKCDpale] ($(current page.north west)-(0,{0.6\paperwidth})$) rectangle (current page.north east);
    \node (graph) at ($(current page.north)-(0,0.3\paperwidth)$) {\includegraphics[width=0.9\paperwidth]{pltAdpSimp.png}};
    \node[below= 0.15cm  of graph] {Figure \arabic{figure}: The errors of the adaptive Simpson Rule with decreasing tolerance, for different functions.};
\end{tikzpicture}

\restoregeometry

\end{document}
